\documentclass[12pt]{article}

%% FONTS
%% To get the default sans serif font in latex, uncomment following line:
 \renewcommand*\familydefault{\sfdefault}
%%
%% to get Arial font as the sans serif font, uncomment following line:
%% \renewcommand{\sfdefault}{phv} % phv is the Arial font
%%
%% to get Helvetica font as the sans serif font, uncomment following line:
% \usepackage{helvet}
\usepackage[small,bf,up]{caption}
\renewcommand{\captionfont}{\footnotesize}
\usepackage[left=1in,right=1in,top=1in,bottom=1in]{geometry}
\usepackage{graphics,epsfig,graphicx,float,subfigure,color}
\usepackage{amsmath,amssymb,amsbsy,amsfonts,amsthm}
\usepackage{url}
\usepackage{boxedminipage}
\usepackage[sf,bf,tiny]{titlesec}
 \usepackage[plainpages=false, colorlinks=true,
   citecolor=blue, filecolor=blue, linkcolor=blue,
   urlcolor=blue]{hyperref}
\usepackage{enumitem}
\usepackage{verbatim}
\usepackage{tikz,pgfplots}

\newcommand{\todo}[1]{\textcolor{red}{#1}}
% see documentation for titlesec package
% \titleformat{\section}{\large \sffamily \bfseries}
\titlelabel{\thetitle.\,\,\,}

\newcommand{\bs}{\boldsymbol}
\newcommand{\alert}[1]{\textcolor{red}{#1}}
\setlength{\emergencystretch}{20pt}

\begin{document}

\begin{center}
  \vspace*{-2cm}
{\small MATH-GA 2012.001 and CSCI-GA 2945.001, Georg Stadler \&
  Dhairya Malhotra (NYU Courant)}
\end{center}
\vspace*{.5cm}
\begin{center}
\large \textbf{%%
Spring 2019: Advanced Topics in Numerical Analysis: \\
High Performance Computing \\
Assignment 5 (due Apr.\ 29, 2019) }
\end{center}

\begin{center}
  \vspace*{0.75cm}
{\Large \textbf{Terrence Alsup}}
\end{center}

% ****************************
\begin{enumerate}
% --------------------------

\item {\bf MPI ring communication.}  

The C++ program {\tt int\_ring.cpp} sends an array of integers around each processor with each processor adding its rank to every element of the array.  We ``loop" over all of the processors {\tt Nrepeat} times.  By the end every element of the array should have the value
\[
\text{\tt Nrepeat} \times \frac{\text{\tt size} \times (\text{\tt size} - 1)}{2}
\]
where {\tt size} is the number of processors.  The program can by run with the command
\[
\text{\tt mpirun -np 4 ./int\_ring}
\]
with 4 being the number of processors.  We loaded the following modules on a CIMS desktop:
\begin{itemize}{}
\item {\tt gcc-8.1}
\item {\tt mpi/openmpi-x86\_64}
\end{itemize}
The program was tested on a CIMS desktop with a Intel(R) Core(TM) i7-6700 CPU @ 3.40GHz processor and 4 cores.  The table below shows the estimated latency and bandwidth for a different number of processors.

\begin{table}[H]
\centering
\begin{tabular}{l | r  r}
{\tt size} & Latency (ms) & Bandwidth (GB/s)\\
\hline
2 & 8.57 e-04 & 1.18 e+00\\
3 & 9.52 e-04 & 7.88 e-01\\
4 & 1.22 e-03 & 5.64 e-01
\end{tabular}
\caption{The estimated latency and bandwidth on a CIMS desktop.  {\tt Nrepeat} = 10000, and the length of the array of integers was $2^{18}$, which has a size of approximately 2MB.}
\label{table:table1}
\end{table}



\item {\bf Details regarding our final project.}

  \begin{center}
  \begin{tabular} {|c|p{9cm}|p{2cm}|}
    \hline
    \multicolumn{3}{|c|}{\bf Project: Parallel KMC} \\
    \hline
    Week & Work & Who  \\ \hline \hline
    04/15-04/21 & Read paper.  Start thinking about
    implementation. & Anya, Terrence \\ \hline
    04/22-04/28 & Write pseudo-code.  Discuss boundary communication and time updating between blocks and sectors.
    Write up weekly plan. & Anya, Terrence \\ \hline
    04/29-05/05 & Implement 1 block for 1D.  Compare to 1D serial and PDE (error for different $L$, $\beta$). Implement multiple blocks and compare.  & Anya (first item), Terrence (third item)  \\ \hline
    05/06-05/12 & Fix 1D bugs.  Check 2D serial.  Start 2D implementation for 1 and multiple blocks.  Compare to PDE. & Anya (second item), Terrence (third item) \\ \hline
    05/13-05/19 & Fix bugs.  Run scalability tests. Work on
    presentation slides and report.  & Anya, Terrence \\ \hline
  \end{tabular}
  \end{center}



\end{enumerate}
\end{document}
